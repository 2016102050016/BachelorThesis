\chapter{%总结
  结论}
\label{cha:summ}

在本工作中,我们实现了一个高质量的三维点云重建系统。
无论输入图像是渲染产生的还是在实际生活中拍摄的,本工作重建质量%都
% 远远高于
%都
均
优于现有算法 PointSetGen\cite{pointsetgen},且用户的可控性更高,更加用户友好。%,用户对于重建过程的可控性也更高。

% 本工作试图解决是问题是三维重建,即从图片中恢复模型的三维结构。
本工作试图解决的重建问题看似应用狭窄,实用性低,但实则不然。
在当今的电影、游戏、动漫等文化产业中,三维建模技术%与后期技术
越来越受到重视。
为了设计出高质量的三维模型,%供后期工作使用,
无数的三维建模师%鞠躬尽瘁
尽心尽力
地坚守在自己的岗位,加班加点、任劳任怨。
诚然,人工建模的速度总是有限的,
% 如果
但是,只要这个项目能够得以扩展和推广,
例如将重建的点云数据进一步转化为三角面片和 CAD 原语,供相关工作者使用,
那么它们的工作效率必将%改变
大幅%度
地提升,同时也会对相关行业产生深远的影响。

% 的问题很简单,但它对于计算机图形学的发展意义重大。在计算机图形学中,
% 三维模型生成与建模一直是一个困难的问题。



必须承认,本工作的成功与近年来%基于
点云%的
三维深度学习的提出和深度生成模型的改进密不可分。
将深度生成模型 VAE/GAN\cite{vaegan} 引入到重建问题后,算法的质量的确有很大的改进。
但是,我们必须指出:单就点云生成而言,
本工作还远远比不上图像生成等任务的前沿进展,如 PG-GAN\cite{pggan} 等。
仔细观察应该不难发现,在本项目的重建结果中,仍然存在少量的噪点、散点等现象。%还是清晰可见的。
因此,这样的重建结果还没有达到“以假乱真”的境界,更不能与前沿工作相提并论。

究其原因,%笔者认为
我们认为
这是由于
点云自身的缺陷导致的。
在第 \ref{section:pcintro} 节中,
我们讨论了点云三维深度学习的劣势以及亟待解决的问题,
如高效合理的局部特征提取%、合理的转置卷积算子
等。
这些问题都是点云三维深度学习的重要课题。它们直接决定了
%深度生成模型中
判别器、编码器的信息提取以及抗过拟合的能力,
% 间接地
影响着点云数据在深度生成模型中的表现。
好比没有卷积神经网络的二维深度学习一样,
只要这些问题%如果不
没有被有效地%加以
解决,
那么点云三维深度学习的发展空间和应用前景必然受限。%是受限的。

现在已经有一些工作,尝试着改进局部信息的提取过程,
如 PointNet++ \cite{pointnet2}、PointCNN \cite{pointcnn}等。
% 但这些工作的提出的方案
%并没有太大的改善
但在本工作相关的实验中,我们并没有观察到这些新工作对于点云生成质量的改善。%在本工作中
关于点云生成质量低下的具体原因及其改进措施,我们将把它作为未来的研究方向与工作目标。

%,因此我们曾花费了大量的篇幅

% 总结